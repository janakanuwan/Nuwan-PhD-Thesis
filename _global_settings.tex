\ifdefined\DoubleSided
  \documentclass[twoside,openright,a4paper]{nusthesis}
\else
  \documentclass[a4paper]{nusthesis}
\fi

\dsp % pseudo double spacing

%%% The abstract of the thesis is titled as "Abstract" by default. 
%%% However, the Graduate Division may require you to name it as "Summary".
\renewcommand{\abstractname}{Summary}

%%% Depth of section numbering
\setcounter{secnumdepth}{3}

%%% Depth of numbering in table of content
\setcounter{tocdepth}{2}

\usepackage{indentfirst} % indent the first paragraph of a section


\usepackage{bookmark}
\usepackage[
  backend=biber,
  style=ieee,
  citestyle=numeric,
  giveninits=true,
  sorting=nyt,
  maxbibnames=99,
  dashed=false,
  doi=false]{biblatex}
\DeclareFieldFormat*{title}{``#1''\newunitpunct} % move comma outside the quotation mark
\addbibresource{chapters/references.bib}

\usepackage{microtype} % Better typography
\usepackage{tabulary} % Automatic table sizing
\usepackage{hhline}

\usepackage{enumitem}

\usepackage{hyperref}
\renewcommand{\sectionautorefname}{\S}
\let\subsectionautorefname\sectionautorefname
\let\subsubsectionautorefname\sectionautorefname
\renewcommand{\chapterautorefname}{Chapter}

\usepackage{listings}
\renewcommand{\lstlistingname}{Code}
\lstset{language=SQL, 
    morekeywords={ONLINE, WITHIN},
    frame=none,
    float,
    basicstyle=\ttfamily\normalsize,
    keywordstyle=\bfseries,
    numbers=none,
    showstringspaces=false,
    aboveskip=-2pt, %\smallskipamount
    belowskip=-4pt, %\smallskipamount
    xleftmargin=1em,
    escapeinside={(*@}{@*)},
    captionpos=b
}

\let\proof\relax
\let\endproof\relax
\usepackage{amsthm}
\theoremstyle{definition}
\newtheorem{defn}{Definition} \newcommand{\defnautorefname}{Definition}
\theoremstyle{plain}
\newtheorem{rul}{Rule} \newcommand{\rulautorefname}{Rule}
\newtheorem{thm}{Theorem} \newcommand{\thmautorefname}{Theorem}
\newtheorem{lemma}{Lemma} \newcommand{\lemmaautorefname}{Lemma}
\newtheorem{coroll}{Corollary} \newcommand{\corollautorefname}{Corollary}

\usepackage{mathtools}
\DeclarePairedDelimiter{\ceil}{\lceil}{\rceil}
\DeclarePairedDelimiter{\floor}{\lfloor}{\rfloor}
\DeclarePairedDelimiter{\vbar}{\vert}{\vert}
\DeclarePairedDelimiter{\vbarbar}{\Vert}{\Vert}
\DeclarePairedDelimiter{\parenLR}{\lparen}{\rparen}
\DeclarePairedDelimiter{\brackLR}{\lbrack}{\rbrack}
\DeclarePairedDelimiter{\braceLR}{\lbrace}{\rbrace}
\DeclarePairedDelimiter{\angleLR}{\langle}{\rangle}
\DeclareMathOperator*{\argmin}{arg\,min}
\DeclareMathOperator*{\argmax}{arg\,max}

\usepackage{interval}
\intervalconfig{soft open fences}
\newcommand{\intervalO}{\interval[open]}
\newcommand{\intervalOL}{\interval[open left]}
\newcommand{\intervalOR}{\interval[open right]}

\usepackage[linesnumbered,ruled,vlined]{algorithm2e}
\SetAlgorithmName{Algorithm}{Algorithm}{Algorithm}
\newcommand{\AlgoFontSize}{\small} % \scriptsize \footnotesize \small \normalsize
\IncMargin{0.5em}
\SetCommentSty{textnormal}
\SetNlSty{}{}{:}
\SetAlgoNlRelativeSize{0}
\SetKwInput{KwGlobal}{Global}
\SetKwInput{KwPrecondition}{Precondition}
\SetKwProg{Proc}{Procedure}{:}{}
\SetKwProg{Func}{Function}{:}{}
\SetKw{And}{and}
\SetKw{Or}{or}
\SetKw{To}{to}
\SetKw{DownTo}{downto}
\SetKw{Break}{break}
\SetKw{Continue}{continue}
\SetKw{SuchThat}{\textit{s.t.}}
\SetKw{WithRespectTo}{\textit{wrt}}
\SetKw{Iff}{\textit{iff.}}
\SetKw{MaxOf}{\textit{max of}}
\SetKw{MinOf}{\textit{min of}}
\SetKwBlock{Match}{match}{}{}

\usepackage{array}
\newcolumntype{L}[1]{>{\raggedright\let\newline\\\arraybackslash\hspace{0pt}}m{#1}}
\newcolumntype{C}[1]{>{\centering\let\newline\\\arraybackslash\hspace{0pt}}m{#1}}
\newcolumntype{R}[1]{>{\raggedleft\let\newline\\\arraybackslash\hspace{0pt}}m{#1}}

\usepackage{color}
\usepackage[usenames,dvipsnames]{xcolor}
\usepackage{soul}
\soulregister\cite7
\soulregister\ref7
\soulregister\pageref7
\soulregister\autoref7
\soulregister\eqref7
\newcommand{\hlred}[2][Lavender]{{\sethlcolor{#1}\hl{#2}}}
\newcommand{\hlblue}[2][SkyBlue]{{\sethlcolor{#1}\hl{#2}}}
\newcommand{\hlyellow}[2][GreenYellow]{{\sethlcolor{#1}\hl{#2}}}
\newcommand{\hlgreen}[2][YellowGreen]{{\sethlcolor{#1}\hl{#2}}}

\let\originaleqref\eqref
\renewcommand{\eqref}{Equation~\ref}


%%% Use this command when you need to add a hyphen with hyphenation 
%%% enabled for the individual compound words
\newcommand{\zz}{-\nolinebreak\hspace{0pt}}

%%% Handy commands for image/figure insertion
\newcommand*{\RootPicDir}{pic}
\newcommand*{\PicDir}{\RootPicDir}
\newcommand*{\ResetPicDir}{\renewcommand*{\PicDir}{\RootPicDir}}
\newcommand*{\SetPicSubDir}[1]{\renewcommand*{\PicDir}{\RootPicDir /#1}}
\newcommand*{\Pic}[1]{\PicDir /#1}

\newcommand*{\RootExpDir}{exp}
\newcommand*{\ExpDir}{\RootExpDir}
\newcommand*{\ResetExpDir}{\renewcommand*{\ExpDir}{\RootExpDir}}
\newcommand*{\SetExpSubDir}[1]{\renewcommand*{\ExpDir}{\RootExpDir /#1}}
\newcommand*{\Exp}[2]{\ExpDir /#2.#1}

\newcommand*{\BeforeCaptionVSpace}{1ex}
\newcommand*{\BeforeSubCaptionVSpace}{0.75ex}

%%% For example code used only
\usepackage{lipsum} % for generating dummy text
\newcommand{\CMD}[1]{\texttt{\string#1}} % for typing a command




%%  supported package
%% https://www.acm.org/publications/taps/whitelist-of-latex-packages
\usepackage{subcaption}
\usepackage{multirow}
\usepackage{xcolor}
\usepackage{enumitem}


\usepackage[nomessages]{fp}% http://ctan.org/pkg/fp (math)
\usepackage{soul} % text highlighting
\usepackage{fp} % floating point


% change autoref formatting
\renewcommand{\sectionautorefname}{Section}
\renewcommand{\subsectionautorefname}{Section}
\renewcommand{\subsubsectionautorefname}{Section}




\ifcomment

\usepackage{silence} % ignore trivial warnings
\WarningFilter{biblatex}{File 'english-ieee.lbx'}

\fi

% custom macros 
\renewcommand{\b}[1]{\textbf{#1}}
\renewcommand{\i}[1]{\textit{#1}}

\ifcomment
%% keep comments
\newcommand{\hide}[1]{}
\newcommand{\note}[1]{\textcolor{blue}{<< #1 >>}}
\renewcommand{\added}[1]{\textcolor[rgb]{0.1, 0.56, 1}{#1}}
\newcommand{\modif}[1]{\textcolor[rgb]{0.5,0.5,0.9}{#1}}
\newcommand{\cut}[1]{\textcolor[rgb]{0.5,0.5,0.5}{CUT: #1}}
\newcommand{\checkme}[1]{\textcolor{red}{CHECKME: #1}}
\newcommand{\todo}[1]{\textcolor{red}{TODO: #1}}
\newcommand{\moved}[1]{\textcolor{green}{#1}}
\renewcommand{\deleted}[1]{\textcolor[rgb]{0.8,0.8,0.8}{#1}}
\newcommand{\removedbutkept}[1]{\textcolor[rgb]{0.8,0.5,0.9}{#1}}
\newcommand{\temporary}[1]{\textcolor[rgb]{0.9,0.9,0.9}{{#1}}}

\newcommand{\nuwan}[1]{\textcolor{blue}{NUWAN: #1}}
\newcommand{\zsd}[1]{\textcolor[rgb]{0.5,0.1,0.8}{ZSD: #1}}

\else
%% remove comments
\newcommand{\hide}[1]{}
\newcommand{\note}[1]{}
\renewcommand{\added}[1]{#1}
\newcommand{\modif}[1]{#1}
\newcommand{\cut}[1]{}
\newcommand{\checkme}[1]{#1}
\newcommand{\todo}[1]{}
\newcommand{\moved}[1]{#1}
\renewcommand{\deleted}[1]{}
\newcommand{\removedbutkept}[1]{#1}
\newcommand{\temporary}[1]{}

\newcommand{\nuwan}[1]{}
\newcommand{\zsd}[1]{}

\fi

%% statistics
\newcommand{\meansd}[2]{$M = #1,\: SD = #2$} 
\newcommand{\meadianiqr}[2]{$Mdn = #1,\: IQR = #2$} 
\newcommand{\range}[2]{$MIN = #1,\: MAX = #2$} 
\newcommand{\kruskalwallis}[3]{Kruskal-Wallis test, $\chi^{2}(#1)\:=\:#2$, #3} 
\newcommand{\mannwhitney}[3]{Mann-Whitney U test, #1, #2} 
\newcommand{\wilcoxon}[2]{$Z=#1, \: p#2$} 
\newcommand{\wilcoxonef}[3]{$Z=#1, \: p#2, \:r=#3$} 
\newcommand{\ttest}[3]{$t(#1)=#2, \: p#3$} 
\newcommand{\ttestef}[4]{$t(#1)=#2, \: p#3, \:d=#4$} 
\newcommand{\spearmancorr}[2]{$Spearman's \: \rho = #1,\: \text{p} = #2$} 
\newcommand{\andersondarling}[2]{\i{Anderson-Darling}$ = #1,\: p = #2$} 

\newcommand{\friedman}[4]{$\chi^{2}(#1) = #2,\: \pval{#3},\: \text{\hide{Kendall’s }W} = #4$} % Friedman test: DF, Chi-square value, p, Kendall's W
\newcommand{\anova}[4]{$F_{#1,#2}=#3$, \pval{#4}} % Classic ANOVA: DF, DFDen, F and p
\newcommand{\anovas}[4]{$F_{#1,#2}=#3$, $p<#4$} % Significant ANOVA: DF, DFDen, F and p
\newcommand{\anovawef}[5]{$F_{#1,#2}=#3$, \pval{#4}, \eff{#5}} % ANOVA with effect size
% \newcommand{\anovaswef}[5]{{($F_{#1,#2}=#3$, $p<#4$, \eff{#5})}} % ANOVA Significant with effect size
\newcommand{\anovaswef}[5]{{$F_{#1,#2}=#3$, $p<#4$}} % ANOVA
\newcommand{\anovawefp}[5]{$F_{#1,#2}=#3$, \pval{#4}, \effp{#5}}
\newcommand{\effp}[1]{$\eta^{2}_{p}=#1$} % ANOVA with partial effect
\newcommand{\eff}[1]{$\eta^2=#1$}
\newcommand{\effd}[1]{$d=#1$}
\newcommand{\effr}[1]{$r=#1$}
\newcommand{\epsiloncorrection}[1]{$\epsilon=#1$}

\newcommand{\pval}[1]{$p#1$}
\newcommand{\pbonf}[1]{$p_{bonf}#1$}

\renewcommand{\quote}[1]{``#1''}
\newcommand{\quoteby}[2]{``#2 (#1)''}

\newcommand{\significantI}[0]{$^{\star}$}
\newcommand{\significantII}[0]{$^{\dagger}$}
\newcommand{\significantIII}[0]{$^{\ddagger}$}


\newcommand{\participantproportion}[2]{\textit{#1/#2 = \FPeval{\result}{round(#1*100/#2,1)}\result\%}} 


\newcommand{\highlight}[1]{\hl{\textbf{#1}}}
\newcommand{\highlightonly}[1]{\colorbox{yellow}{#1}}

% inline image
\newcommand{\inlineimg}[1]{%
  \begingroup\normalfont
  \includegraphics[width=2\fontcharht\font`\B]{#1}%
  \endgroup
}

% ref: https://tex.stackexchange.com/questions/346211/partially-colored-cells
\newlength\maxlen
% \databar{max_value}{current_value} (absolute bar)
\newcommand\databar[3][blue!30]{%  
  \FPeval\result{round(#3/#2:4)}%
  \rlap{\textcolor{#1}{\hspace*{\dimexpr-\tabcolsep+.5\arrayrulewidth}%
        \rule[-.05\ht\strutbox]{\result\maxlen}{.95\ht\strutbox}}}%
  \makebox[\dimexpr\maxlen-2\tabcolsep+\arrayrulewidth][r]{#3}}
% \databarrel{max_value}{min_vlaue}{current_value} (relative bar)
\newcommand\databarrel[4][blue!30]{%
  \FPeval\result{round((#4-#3)/(#2-#3):4)}%
  \rlap{\textcolor{#1}{\hspace*{\dimexpr-\tabcolsep+.5\arrayrulewidth}%
        \rule[-.05\ht\strutbox]{\result\maxlen}{.95\ht\strutbox}}}%
  \makebox[\dimexpr\maxlen-2\tabcolsep+\arrayrulewidth][r]{#4}}
\def\databarlength{xx.xx} % 4 digits
\settowidth\maxlen{\databarlength}
\addtolength\maxlen{\dimexpr2\tabcolsep-\arrayrulewidth}

% studies
\newcommand{\studyone}[1]{\textit{study 1#1}}
\newcommand{\studytwo}[0]{\textit{study 2}}
\newcommand{\studythree}[0]{\textit{study 3}}
\newcommand{\studyfour}[0]{\textit{study 4}}

\newcommand{\Studyone}[1]{\textit{Study 1#1}}
\newcommand{\Studytwo}[0]{\textit{Study 2}}
\newcommand{\Studythree}[0]{\textit{Study 3}}
\newcommand{\Studyfour}[0]{\textit{Study 4}}

\newcommand{\factor}[1]{\textbf{\textit{#1}}.}


% common DVs
\newcommand{\dprime}[0]{$d^{'}$} 
\newcommand{\hitrate}[0]{\textit{H}}
\newcommand{\falsealarmrate}[0]{\textit{F}}
\newcommand{\reactionTime}[0]{\textit{RT}}
\newcommand{\immediateRecall}[0]{\textit{Recall Accuracy}}

\newcommand{\readingTime}[0]{\textit{Reading Time}}
\newcommand{\readingAccuracy}[0]{\textit{Reading Accuracy}}
\newcommand{\adjustedReadingAccuracy}[0]{\textit{Adjusted Reading Accuracy}}
\newcommand{\notificationAccuracy}[0]{\textit{Notification Accuracy}}

\newcommand{\SUS}[0]{\textit{SUS}}
\newcommand{\perceivedTaskLoad}[0]{\textit{RTLX}}
\newcommand{\perceivedInterruption}[0]{\textit{Perceived Interruption}}

\newcommand{\noticeability}[0]{\textit{Noticeability}}
\newcommand{\understandability}[0]{\textit{Understandability}}
\newcommand{\perceivedEaseIdentification}[0]{\textit{Ease of Identification}}
\newcommand{\comfortability}[0]{\textit{Comfortability}}
\newcommand{\perceivedEffectiveness}[0]{\textit{Perceived Effectiveness}}

\newcommand{\preference}[0]{\textit{Preference}}

\newcommand{\Interruption}[0]{\textit{Interruption}}
\newcommand{\Reaction}[0]{\textit{Reaction}}
\newcommand{\Comprehension}[0]{\textit{Comprehension}}
\newcommand{\Satisfaction}[0]{\textit{Satisfaction}}

\newcommand{\RQMainThesis}[0]{\textit{How do we modify the information presentation of visual OHMD notifications to minimize attention costs while maintaining communicative effectiveness during multitasking?}}
\newcommand{\RQMainProgressBar}[0]{\textit{How can we distribute notification content to engage different vision regions to minimize the attention costs of OHMD notifications?}}
\newcommand{\RQMainIconNotif}[0]{\textit{How do we convert notification content to easily recognizable shapes to minimize the attention costs of OHMD notifications?}}
\newcommand{\RQMainGradNotif}[0]{\textit{How can we control the luminance of notification content to minimize the attention costs of OHMD notifications?}}

% individual chapters


% study 1
\newcommand{\receiver}[0]{\textit{wearer}}
\newcommand{\Receiver}[0]{\textit{Wearer}}
\newcommand{\observer}[0]{\textit{non-wearer}}
\newcommand{\Observer}[0]{\textit{Non-wearer}}
\newcommand{\prefixReceiver}[0]{[W]}
\newcommand{\prefixObserver}[0]{[N]}


% IVs
\newcommand{\type}[0]{\textit{type}}
\newcommand{\linearbar}[0]{\textit{linear bar}}
\newcommand{\Linearbar}[0]{\textit{Linear bar}}
\newcommand{\circularbar}[0]{\textit{circular bar}}
\newcommand{\Circularbar}[0]{\textit{Circular bar}}
\renewcommand{\textbar}[0]{\textit{text label}}
\newcommand{\Textbar}[0]{\textit{Text label}}

\newcommand{\persistence}[0]{\textit{persistence}}
\newcommand{\continuous}[0]{\textit{continuous}}
\newcommand{\Continuous}[0]{\textit{Continuous}}
\newcommand{\intermittent}[0]{\textit{intermittent}}
\newcommand{\Intermittent}[0]{\textit{Intermittent}}



% DVs
\newcommand{\progressAccuracy}[0]{\textit{Progress Accuracy}}
\newcommand{\progressBarFocus}[0]{\textit{Progress Focus \%}}

\newcommand{\progressIdentificationError}[0]{\textit{Progress Error}}
\newcommand{\distractionDigreee}[0]{\textit{Degree of Distraction}}
\newcommand{\gazePercentage}[0]{\textit{Gaze Percentage}}


% IVs
\newcommand{\primaryinfo}[0]{\textit{primary info}}
\newcommand{\secondaryinfo}[0]{\textit{secondary info}}

\newcommand{\format}[0]{\textit{format}}
\newcommand{\textformat}[0]{\textit{text format}}
\newcommand{\Textformat}[0]{\textit{Text format}}
\newcommand{\iconformat}[0]{\textit{pictogram format}}
\newcommand{\Iconformat}[0]{\textit{Pictogram format}}

\newcommand{\complexity}[0]{\textit{density}}
\newcommand{\complexities}[0]{\textit{densities}}
\newcommand{\encodingcomplexity}[0]{encoding density}
\newcommand{\Encodingcomplexity}[0]{Encoding density}
\newcommand{\single}[0]{\textit{one}}
\newcommand{\Single}[0]{\textit{One}}
\newcommand{\multi}[0]{\textit{two}}
\newcommand{\Multi}[0]{\textit{Two}}
\newcommand{\singlecomplexity}[0]{\complexity{} \single{}}
\newcommand{\multicomplexity}[0]{\complexity{} \multi{}}

\newcommand{\textnotif}[1]{\textit{text notification#1}}
\newcommand{\Textnotif}[1]{\textit{Text notification#1}}
\newcommand{\iconnotif}[1]{\textit{icon-augmented notification#1}}
\newcommand{\Iconnotif}[1]{\textit{Icon-augmented notification#1}}
\newcommand{\nonotif}[0]{\textit{no-notification}} 
\newcommand{\Nonotif}[0]{\textit{No-notification}} 

\newcommand{\iconsingle}[0]{$icon_{\single{}}$}
\newcommand{\iconmulti}[0]{$icon_{\multi{}}$}
\newcommand{\textsingle}[0]{$text_{\single{}}$}
\newcommand{\textmulti}[0]{$text_{\multi{}}$}

\newcommand{\tasknav}[0]{\textit{Navigation}}
\newcommand{\taskbrowse}[0]{\textit{Browsing}}


% DVs



\newcommand{\iconname}[1]{$<$#1 icon$>$}

\newcommand{\iconsize}[0]{8mm}
\newcommand{\iconsizeLarge}[0]{12mm}

\newcommand{\iconsetimage}[1]{%
  \includegraphics[height=\iconsize{}]{#1}
}

% image sources
\newcommand{\materialIcons}[0]{\href{https://fonts.google.com/icons?selected=Material+Icons}{Google Material Icons}}
\newcommand{\flatIcons}[0]{\href{https://www.flaticon.com}{Flaticon website (premium license)}}
\newcommand{\nounProject}[0]{\href{https://thenounproject.com}{The Noun Project}}








% common
\newcommand{\fadeduration}[1]{\textit{fade-duration#1}}

\newcommand{\Fading}[0]{{Fade-in}}
\newcommand{\fading}[0]{{fade-in}}

% IVs
\newcommand{\animation}[1]{\textit{animation#1}}
\newcommand{\instant}[0]{\textit{blast}}
\newcommand{\slowfade}[0]{\textit{slow-fade}}
\newcommand{\fastfade}[0]{\textit{fast-fade}}
\newcommand{\scroll}[0]{\textit{scroll}}

\newcommand{\Animation}[1]{\textit{Animation#1}}
\newcommand{\Instant}[0]{\textit{Blast}}
\newcommand{\Slowfade}[0]{\textit{Slow-fade}}
\newcommand{\Fastfade}[0]{\textit{Fast-fade}}
\newcommand{\Scroll}[0]{\textit{Scroll}}


% study 1
\newcommand{\location}[1]{\textit{location#1}}
\newcommand{\desktop}[0]{\textit{diff-depth}}
\newcommand{\glass}[0]{\textit{same-depth}}

% study 2
\newcommand{\mobility}[0]{\textit{mobility}}
\newcommand{\sitting}[0]{\textit{sitting}}
\newcommand{\walking}[0]{\textit{walking}}
